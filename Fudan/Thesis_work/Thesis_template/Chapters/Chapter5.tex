% Chapter 5

\chapter{Summary and Future Directions} % Main chapter title

\label{Chapter5} % For referencing the chapter elsewhere, use \ref{Chapter3} 

\lhead{Chapter 5. \emph{Summary and Future  Directions}} % This is for the header on each page - perhaps a shortened title

%----------------------------------------------------------------------------------------
\section{Summary}
      To summarize, we started with a brief introduction to category theory and some mathematical structures like Algebras, Bialgebras and Hopf Algebras which were used to construct the 
Drinfeld Double of a group. We then introduced the lattice models which are used to classify the topological phases of matter. Kitaev Quantum Double Models were introduced along with some
properties of the system like ground state construction, ribbon operator construction leading to detection of excitations given a particular group. Construction of ribbon operators in a Quantum 
Double with boundaries leading to detection of condensation of excitations on boundaries is presented. SageMath \citep{Reference9} has been used to compute the ribbon operators on lattice with boundaries, excitations
which condense on the boundary, the results for $S_{3}$ have been presented in thesis and for any generalized group the methods to compute the same have been presented in the Appendix \ref{AppendixA}. An introduction
to Levin-Wen models with and without boundaries, with a brief introduction to excitation detection and string operator that is the ribbon operator equivalent have been presented. Statements relating excitations in 
the bulk, excitations on the boundary and excitations in the condensed phase have been presented in one step as well as two step condensation. In the end, an attempt has been made to show Quantum Doubles
as a subclass of Levin-Wen models and various results have been verified in the case of the Toric Code and $D(S_{3})$.  
\pagebreak
\section{Future Directions}
      Some of the following ideas require further work :
\begin{itemize}
 \item Computing the commutation relationship equivalent of ribbon operators for string operators. Solving for string operators in the presence of boundary using the commutation relationship. For the 
       $Q_{v}$ operator, the result has been computed but $B_{p}$ operator is still a work in progress. Once the string operators in a lattice with boundaries are computed, the next step is to compute
       the ground state which would lead to the computation of topological entanglement entropy.
 \item Construction of ribbon operators in higher dimensions \citep{Reference10}.
 \item Construction of boundary conditions for Twisted Quantum Double models in 2D \citep{Reference11, Reference12} and 3D \citep{Reference13}, and extending the idea of ribbon operators for these models with boundaries.
 \item In the case of $D(S_{3})$, construction of condensable algebras play an important role as these would help in understanding the computation of boundary excitations, condensed phase and excitations which condense
       on the boundary.
 \item Topological Entanglement Entropy in the presence of ribbon operators in the case of $D(S_{3})$
 \item Given the center of a UTC which is a MTC, one can find many UTC's which are equivalent upto Morita equivalence. Given a UTC, one can construct the boundary labels as left modules of the UTC. Therefore, 
       given a MTC, is it possible to predict the boundary labels.
 \item Understanding of co-dimension 2 boundaries and application to the case of $D(S_{3})$.
 \item The ribbon operators computed for $S_{3}$ do not provide the required insight on splitting of excitation on boundary. In the sense the action of the ribbon operator on a cylinderical lattice 
       connecting boundaries at the top and bottom, is equivalent to the horizontal surface action. The construction of ribbon operator with vertical surface action might provide an insight into the 
       splitting of excitation on boundary.
 
\end{itemize}
