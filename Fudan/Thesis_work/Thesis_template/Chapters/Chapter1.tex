% Chapter 1

\chapter{Mathematical Preliminaries} % Main chapter title

\label{Chapter1} % For referencing the chapter elsewhere, use \ref{Chapter1} 

\lhead{Chapter 1. \emph{Mathematical Preliminaries}} % This is for the header on each page - perhaps a shortened title

%----------------------------------------------------------------------------------------

\section{Category Theory : Definitions and Structures}
 The aim of the section is to provide an overview of the various definitions and formalisms which will be used later in the presentation.
To begin with Linearity, Semisimplicity, Finiteness are defined and then move onto Monoidal Categories,the later sections  
assume Monoidal Categories as a base to introduce other structures like Braiding, Rigidity and Twist, leading to 
the definition of Modularity, leading to Modular Tensor Categories. 

%----------------------------------------------------------------------------------------

\subsection{Linearity, Semisimplicity, Finiteness}

\begin{defn}
Linearity :\\
          A category C is said to be linear, if the Homset(A,B) i.e., the set of morphisms from A to B, $\forall$ A,B $\in$ C forms a vector space over a field of charecteristic zero.
\end{defn}

\begin{defn}
Simple  :\\ 
	An object $x$ $\in$ a category C is said to be simple if $Hom(x,x)$ $\cong$ $C$ (the set of complex numbers). That is the endomorphisms of $x$ is equivalent to $C$. \\
	For further insight refer \citep{Reference15}.
\end{defn}

\begin{defn}
Semisimplicity :\\
          A category C is said to be semisimple, if  every object in C can be written as a direct sum of simple objects in C.
\end{defn}

\begin{defn}
Finite :\\
      A category C is said to be finite, if the number of simple objects in C is finite.
\end{defn}

\subsection{Modular Tensor Categories}
\begin{defn}
Monoidal Category: \\
A category $M$ is said to be monoidal if it is equipped with the following structure :\\
\begin{enumerate}
\item A functor called the tensor product $\otimes$ : $\xymatrix@1{ M\times M}$ $\longrightarrow$ $M$ where $\otimes$ $(x,y)$ = $x$ $\otimes$ $y$ 
   and \\ $\otimes$ $(f,g)$ = $f$ $\otimes$ $g$ $\forall$ objects $x$,$y$ $\in$ $M$ and $\forall$ morphisms $f$,$g$ in $M$ \\
   
\item Natural Isomorphisms called the assosciator:\\
                     \begin{center}
                       $\alpha_{x,y,z} : (x \otimes y) \otimes z \longrightarrow x \otimes (y \otimes z)$
                     \end{center}
         The left unitor:
                     \begin{center}
		      $\beta_{x} : 1 \otimes x \longrightarrow x$
		    \end{center}
         The right unitor:
		    \begin{center}
		     $\gamma_{x} : x \otimes 1 \longrightarrow x$
		    \end{center}
         such that the following diagrams commute. 
                    \begin{center} 
                    $\xymatrix { 
		     &((x \otimes y) \otimes z) \otimes w \ar[dr]^{\alpha_{x \otimes y, w, z}}  \ar[dl]^{\alpha_{x,y,z \otimes 1_{w}}}\\ 
		     (x \otimes (y \otimes z)) \otimes w \ar[d]^{\alpha_{x,y \otimes z, w}}        &          & (x \otimes y) \otimes (z \otimes w) \ar[d]^{\alpha_{x,y,z \otimes w}} \\
		     x \otimes (( y \otimes z) \otimes w) \ar@{-}[r]  &\ar[r]^<(0.05){1_{x} \otimes \alpha_{y,z,w}}   & x \otimes (y \otimes ( z \otimes w))
                    }$  
                       \end{center}
          and 
		    \begin{center}
		    $\xymatrix {
		    (x \otimes 1) \otimes y \ar[dr]^{\gamma_{x} \otimes 1_{y}} \ar@{-}[r] & \ar[r]^<(0.05){\alpha_{x,1,y}}   & x \otimes (1 \otimes y) \ar[dl]^{1_{x} \otimes \beta_{y}}\\
		                                         & x \otimes y     
		    }$
		    \end{center}

\end{enumerate}
\end{defn}

\begin{defn}
Rigidity :\\
      A monoidal category $M$ is said to be rigid, if for each object $A$ $\in$ $M$, $\exists$ an object $A^{*}$ $\in$ $M$ together with the maps : 
      \begin{center}
      $i_{A}$ : $1$ $\rightarrow$ $A$ $\otimes$ $A^{*}$,  \\
      $e_{A}$ : $A^{*}$ $\otimes$ $A$ $\rightarrow$ $1$
      \end{center}
\end{defn}

\begin{defn}
    Fusion Category :\\
	    Fusion category is a finite semisimple C-linear rigid monoidal category such that monoidal unit is simple.
\end{defn}

\begin{defn}
Braided Fusion Category :\\
      A braided fusion category $M$ is a fusion category along with the 
following isomorphisms 
\begin{center}
	  $\xymatrix{
	    \sigma_{x,y} : x \otimes y \longrightarrow^{~} y \otimes x
	   }$
	  \end{center} where $x$, $y$ $\in$ $M$
such that the following diagrams commute.
	\begin{center}
	$\xymatrix{
	x \otimes 1 \ar[r]^{\beta_{x}}  \ar[d]^{\sigma_{1,x}} & x \\
	1 \otimes x \ar[ur]_{\gamma_{x}} &
	}$
      \end{center}
and 
      \begin{center}
      $\xymatrix{
        &x\otimes (y \otimes z) \ar[r]^{\sigma_{x,y \otimes z}} &(y \otimes z) \otimes x \ar[dr]^{\alpha_{y,z,x}} & \\
        (x \otimes y) \otimes z \ar[ur]^{\alpha_{x,y,z}} \ar[dr]^{\sigma_{x,y \otimes 1_{z}}}    & & &y \otimes (z \otimes x) \\
	&(y \otimes x) \otimes z \ar[r]^{\alpha_{x,y,z}} & y \otimes (x \otimes z) \ar[ur]^{1_{y} \otimes \sigma_{x,z}} & 
      }$
      \end{center}
\end{defn}

% \par
% Putting together all the structures until now, we can construct a \emph{Finite Semisimple $C$-Linear Rigid Braided Monoidal Category}. We now define
% a \textit{pivotal structure} on a rigid braided monoidal category, which would be further used in the definition of Ribbon Categories (which are Rigid Braided Monoidal Categoires with a twist).
% \par

\begin{defn}
Pivotial Structure on a fusion category is an isomorphism, \\
\begin{center}
  $\delta_{A}$ : $A$ $\rightarrow$ $A^{**}$ ,such that \\
  $\delta_{A \otimes B}$ = $\delta_{A}$ $\otimes$ $\delta_{B}$ \\
  $\delta_{1}$ = $1$
\end{center}
\end{defn}

\begin{defn}
Ribbon Fusion Category is a Braided Fusion Category with a pivotal structure which is compatible with braiding. 
\begin{center}
  $\theta_{A}$ = $\gamma_{A}$ $\circ$ $\delta_{A}$ : $A$ $\rightarrow$ $A$ via $A^{**}$
\end{center}
\end{defn}

\par
Note : Braided Fusion Category with a pivotal structure is not always a Ribbon Fusion Category, as the pivtoal structure must be compatible with braiding.  
\par

% \begin{defn}
% Ribbon Category :\\
% Ribbon Category is a Rigid Braided Monodial Category with a compatible twist, defined by the isomorphism
% \begin{center}
%  $\theta_{A}$ = $\gamma_{A}$ $\circ$ $\delta_{A}$ : $A$ $\rightarrow$ $A$ via $A^{**}$
% \end{center}
% \end{defn}
\begin{defn}
Spherical Fusion Category : \\
Let $f$ $\in$ $End(X)$ where $X$ is a simple object in $C$ i.e., $f$ : $X$ $\longrightarrow$ $X$ then right trace of the map $f$ is given by
\begin{center}
$Tr^{r}(f)$ = $1$ $\rightarrow$ $x$ $\otimes$ $x^{*}$ $\rightarrow$ $x^{**}$ $\otimes$ $x^{*}$ $\rightarrow$ $x$ $\otimes$ $x^{*}$ $\rightarrow$ $1$
\end{center}
On similar lines,
\begin{center}
$Tr^{l}(f)$ = $1$ $\rightarrow$ $x^{*}$ $\otimes$ $x$ $\rightarrow$ $x^{*}$ $\otimes$ $x^{**}$ $\rightarrow$ $x^{*}$ $\otimes$ $x$ $\rightarrow$ $1$
\end{center}
Fusion Category is said to be spherical if $Tr^{l}(f)$ = $Tr^{r}(f)$ $\forall$ $f$ $\in$ C
\end{defn}

\begin{defn}
Quantum Dimension of an object $X$ in category C is given by $d_{X}$ := $Tr(1_{X})$,
we define $D$ = $\varSigma$ $d_{i}^{2}$
\end{defn}

\begin{defn}
We define the modular matrix S, whose elements are given by
\begin{center}
$s_{ij}$ = $Tr(1_{i \otimes j})$ = $d_{i \otimes j}$, $\forall$ $i$, $j$ $\in$ C
\end{center}
\end{defn}
Finally we define Modular Tensor Category (MTC), 
\begin{defn}
Modular Tensor Category :\\
  MTC is a RFC such that the $det(S)$ $\neq$ $0$.
  
  % MTC is a Finite Semisimple Linear Ribbon category along with the modularity condition, that is the S-matrix is invertible (invariant of the hopf link)
% WIP.
\end{defn}

\subsection{Left, Right Modules over a Category}
We define a left module over a tensor category, this is later used to define the boundaries for the Levin-Wen Models in Chapter \ref{Chapter3}.

\begin{defn}
Left Module Category:\\
      Left Module Category over a monoidal category $C$, is a category $M$ equipped with a $C$ action : a functor $\otimes : C \otimes M \rightarrow M$ such that
there are isomorphisms :
\begin{center}
 $ X \otimes (Y \otimes M') \rightarrow (X \otimes Y) \otimes M' $\\
 $ \textbf{1} \otimes M' \rightarrow M'$
\end{center}
for $X, Y \in C$ and $M' \in M$ satisfying some coherence conditions.
\end{defn}

The definition of Right module is on similar lines.
% \subsection{Graphical Calculus}
 % Graphical Calculus is another way of viewing the objects and morphism space of a Category in terms of graphs. The objects are denoted by strings and the morphisms as nodes.
% For example : A $\otimes$ B $\rightarrow$ C $\otimes$ D. 

% Given objects A,B,C,D,E the basis space construction looks as follows and the conversion between the basis space, given by

% is given by F-Matrix (we refer to this as F-move).

% The unit object and unit morphisms are represented by emptiness.

% Give an example citing the above and also a general representation.

% This would be further used in Chapter 4.


\section{Drinfeld Double of a Group}
    To introduce the Drinfeld Double of a Group, denoted by $D(G)$, we first define Algebras, Co-algebras, Bialgebras and Hopf Algebras.
    
\subsection{Algebras, Co-algebras, Hopf Algebras}
\begin{defn}
Algebra : \\
        Let $A$ be a vector space over a field $K$. The triple $(A, m, \eta)$ is an associative algebra, where

    $m : A \otimes A \rightarrow A$ (multiplication map), \\
    $\eta : K \rightarrow A$ (unit map),\\
such that $m, \eta$ satisfy the following commutation diagrams \\
\begin{center}
$\xymatrix{
A \otimes A \otimes A \ar[r]^{m \circ id_{A}} \ar[d]^{id_{A} \circ m} & A \otimes A \ar[d]^{m} \\
A \otimes A \ar[r]^{m}  & A 
}$
\end{center}
and \\
\begin{center}
$\xymatrix{
A \ar[r]^{id_{A} \circ \eta} \ar[d]^{\eta \circ A} & A \otimes A \ar[d]^{m} \\
A \otimes A \ar[r]^{m}  & A 
}$
\end{center} 
\end{defn}

\begin{defn}
Co-Algebra : \\
        Let $C$ be a vector space over a field $K$. The triple $(C, n, \delta)$ is a co-algebra, where

    $n : C \rightarrow C \otimes C$ (comultiplication map)\\
    $\delta : C \rightarrow K$ (counit map),\\
such that $m, \delta$ satisfy the following commutation diagrams \\
\begin{center}
$\xymatrix{
C \ar[r]^{n} \ar[d]^{n} & C \otimes C \ar[d]^{id_{C} \circ n} \\
C \otimes C \ar[r]^{n \circ id_{C}}  & C \otimes C \otimes C
}$
\end{center}
and \\
\begin{center}
$\xymatrix{
C \ar[r]^{\delta} \ar[d]^{\delta} & C \otimes C \ar[d]^{id_{C} \otimes \delta} \\
C \otimes C \ar[r]^{\delta \otimes id_{C}}  & C 
}$
\end{center} 
\end{defn}

\begin{defn}
Bialgebra : \\
    Let $B$ be a vector space over a field $K$. A quintuple $(B, m, n, \eta, \delta)$ is a bialgebra, where
$(B,m,\eta)$ is an algebra and $(B,n,\delta)$ is a co-algebra.
\end{defn}


\begin{defn}
Hopf Algebra: \\
      Let $H$ be a vector space over a field $K$. The quintuple $(H, m, n, \eta, \delta)$ along with the antipode map $S$
\begin{center}
    $S : H \rightarrow H$ 
\end{center}
such that the following diagram commutes 
\begin{center}
 $\xymatrix{
 H \otimes H \ar[rr]^{id_{H} \otimes S} & & H \otimes H \ar[d]^{m} \\
 H \ar[r]^{\delta} \ar[u]^{n} \ar[d]^{n} & K \ar[r]^{\eta} & H \\
 H \otimes H \ar[rr]^{S \otimes id_{H}} &  & H \otimes H \ar[u]^{m}
 }$
\end{center}
forms a Hopf Algebra.
\end{defn}

\subsection{Drinfeld Double (Quantum Double) of a finite group}
    Let $G$ be a finite group and $K$ be a field. The group algebra $K[G]$ is the set of all linear combinations of elements from $G$ with 
the scalars coming from the field $K$. This forms a vector space and by defining the comultiplication $n(g)$ := $g \otimes g$ and 
counit by $\delta(g)$ = 1, and the antipode $S(g) = g^{-1}$ turns the group algebra $K[G]$ into a Hopf Algebra by the above definitions.

    Let $K(G)$ be the set of functions on $G$ with values in $K$. The basis space is given by $\delta_{g}$ defined by the projection 
$\delta_{g}$(h) = $\delta_{g,h}$. This forms a vector space and is an algebra with point-wise multiplication, with the unit given by
$\eta : K \rightarrow K(G)$ is defined by $\eta(\lambda)(g) = \lambda$, with comultiplication, counit and antipode given by :
\begin{center}
    $(n f)(g,h) = f(gh), \delta(f) = f(e), (Sf)(g) = f(g^{-1})$
\end{center}
Thus, $K(G)$ is a Hopf Algebra using the above definitions.

\begin{defn}
Quantum Double of a group $G$, $D(G)$ : \\
    The vector space $K(G) \otimes K[G]$ along with the following structure :\\
\begin{center}
    $(\delta_{g} \otimes x)(\delta_{h} \otimes y) = \delta_{gx,xh}(\delta_{g} \otimes xy)$,\\
    $ 1 = \sum_{g \in G} \delta_{g} \otimes e$, \\
    $ n (\delta_{g} \otimes x) = \sum_{g_{1}g_{2} = g}(\delta_{g_{1}} \otimes x) \otimes (\delta_{g_{2}} \otimes x)$, \\
    $ \Delta(\delta_{g} \otimes x) = \delta_{g,e}$,\\
    $ S(\delta_{g} \otimes x) = \delta_{x^{-1}g^{-1}x} \otimes x^{-1}$.
\end{center}
\end{defn}
forms a Hopf Algebra, which is defined as Quantum Doulbe of a group $G$, $D(G)$.    

\subsection{Representations of Drinfeld Double of a Group}
    Consider an element $a \in G$ and let $\pi$ be a representation of $Z(a)$ over the vector space $W$ 
with basis $\{w_{1}, ...., w_{d}\}$. Define the vector space $V_{\bar{a}, \pi}$ with the basis 
$\{ |b, w_{i}\textrangle : b \in \bar{a}, 1 \leq i \leq d\}$. $V_{\bar{a}, \pi}$ is a representation of $D(G)$ as follows.
For any $b \in a$ fix $k_{b} \in G$ such that $b = k_{b}ak_{b}^{-1}$. Observe that $k_{gbg^{-1}}^{-1}gk_{b}$ is always
in $Z(a)$, for any $w \in W$, $b \in \bar{a}$, and $gh^{*} \in  D(G)$ define
\begin{center}
 $gh^{*}|b,w\textrangle = \delta_{h,b} |gbg^{-1}, \pi(k_{gbg^{-1}}^{-1}gk_{b})w\textrangle$
\end{center}
The above action gives a representation of $D(G)$, the character of this representation is given by,\\
\begin{center}
  $\chi_{(\bar{a}, \pi)}(gh^{*}) = \delta_{h \in \bar{a}}\delta_{gh,hg} tr_{\pi}(k_{h}^{-1}gk_{h})$
\end{center}

All the irreducible representations of the $D(G)$ are indexed by the irreducible representations of the centralizer of the conjugacy classes. 
For more detailed treatment refer to \citep{Reference1, Reference14}

\section{Algebras, Left (Right), BiModules over Algebras in a Category}
    
    The section aims to present the definitions of Algebras in a Category, Left(Right), BiModules over Algebras in
a category. These would be used later to outline the boundary excitations as presented by Kong \cite{Reference5} :

\begin{defn}
Algebra : \\
    Let $A$ be an object in a category $C$. An algebra is a triple $(A, m, \eta)$ where 

    $m : A \otimes A \rightarrow A$,  
    $\eta : K \rightarrow A$, where $K$ is a simple object.\\
such that $m, \eta$ satisfy the following commutation diagrams \\
\begin{center}
$\xymatrix{
A \otimes A \otimes A \ar[r]^{m \circ id_{A}} \ar[d]^{id_{A} \circ m} & A \otimes A \ar[d]^{m} \\
A \otimes A \ar[r]^{m}  & A 
}$
$\xymatrix{
A \ar[r]^{id_{A} \circ \eta} \ar[d]^{\eta \circ A} & A \otimes A \ar[d]^{m} \\
A \otimes A \ar[r]^{m}  & A 
}$
\end{center}
\end{defn}

Given algebra $A$ in category $C$, the right A-module is given by the following : \\
\begin{defn}
Right A-module :\\
    The right module of an algebra $A$ has objects as pairs $(M, \rho_{M})$ where $M \in C$, and $\rho_{M}$ is given by
$\rho_{M} : M \otimes A \rightarrow M$, such that the following commutation diagram is satisfied:\\
\begin{center}
$\xymatrix{
M \otimes A \otimes A \ar[r]^{\rho_{M} \circ id_{A}} \ar[d]^{id_{M} \circ m} & M \otimes A \ar[d]^{\rho_{M}} \\
M \otimes A \ar[r]^{\rho_{M}}  & A 
}$
\end{center}
\end{defn}

Similarly we define the left module, a A-bimodule $M$ is a triple equipped with both $\rho^{l}_{M}$ and $\rho^{r}_{M}$. \\
\begin{defn}
 Commutative Algebra:\\
         Algebra $(A, m, \eta)$ in $C$ is said to be commutative if there exists a natural transformation $C_{M,A}$ given by
         \begin{center}
             $C_{M,A} :  M \otimes A \rightarrow A \otimes M$, where $M \in C$
         \end{center}
\end{defn}

\begin{defn}
 Separable Algebra :\\
 Algebra $(A, m, \eta)$ in $C$ is called separable if there exists a bimodule map $e : A \rightarrow A \otimes A$ such that  
$ m \circ e = id_{A}$. A separable algebra is called connected if $dim(hom(1,A)) = 1)$
\end{defn}

\begin{defn}
 Local Module over a commutative Algebra:\\
      Let $A$ be a commutative Algebra in $C$. Let $(M, \mu_{M})$ be a right A-module. $(M, \mu_{M})$ is called local if the following
      commutation diagram holds :
      \begin{center}
	    $\xymatrix{
	    A \otimes M \ar[r]^{\mu_{M}} \ar[d]^{C_{A,M}} & M \\
	    M \otimes A \ar[r]^{C_{M,A}} & A \otimes M \ar[u]^{\mu_{M}} 
	    }$
      \end{center}
\end{defn}
